\documentclass[11pt]{article}

\usepackage{common}
\usepackage{booktabs}
\title{Practical 1: Regression \\ Predicting the Efficiency of Organic Photovoltaics}
\author{Antonio (email, Camelot.ai username) \\
	Fangli (email, Camelot.ai username) \\
	Xihan (xihanzhang@hsph.harvard.edu, Camelot.ai Xihan)}
\begin{document}


\maketitle{}


%\noindent This is the template you should use for submitting your practical assignments. A full-credit assignment will go through each of these points and provide a thoughtful clear answer.  Note that the limit for the report is 4 pages, please prioritize quality over quantity in your submission.

\section{Technical Approach}

We began by performing exploratory analysis on the sample dataset we were given, which consisted of a training set of 1 million molecules with 256 binary predictors, in addition to their SMILES string and the HOMO--LUMO gap we were seeking to predict. The test set consisted of 824,230 molecules with the same set of predictors as well as their SMILES string. The sample code we were given implemented a default Linear Regression model on the full set of 256 predictors, yielding an $R_\textrm{LR}^2 = 0.461 (\textrm{MSE}_\textrm{LR} = 0.089)$ over the full training set, and a default Random Forest regression with $R_\textrm{RF}^2 = 0.554 (\textrm{MSE}_\textrm{RF} = 0.074)$.

Initial inspection found that out of 256 molecular features, 221 of them were unexpressed (i.e. had $x_i = 0$) for \emph{all} molecules, both in the training set and the test set. Dropping unimportant features would normally call for $K$-fold cross-validation across the training set to ensure those features are consistently unimportant in all cases. However, in the case of null values of certain features for every element of the training set, it is not only legitimate but necessary to drop those features from all further analysis, as fitting along null dimensions would constitute a form of overfitting.

Having reduced the sample dataset to 31 expressed molecular features, we performed both regularized and non-regularized linear regression with cross-validation\footnote{After initially trying 3-fold, 5-fold and 10-fold cross-validation, we settled on 5-fold cross-validation as the best compromise between accuracy and computational speed.} and hyperparameter tuning, under the following methods:

\begin{enumerate}

\item Non-regularized linear regression: Yields $R^2 =  ...$
\item Ridge Regression: 
\item Lasso:
\item Elastic Net:

\end{enumerate}

The results from linear regression on the sample dataset consistently gave us low predictive value, even when compared to generic random forest regression, which led us into pursuing 2 parallel tracks:

\begin{enumerate}

\item Feature engineering: \emph{(Fangli)}

\item Non-linear methods: (Xihan, Antonio)

Among these we concentrated on 2 broad categories:

\begin{enumerate}

\item Tree-based Ensemble methods: Given all the predictors are binary, which is coincide with the branch-like structure of decision tree--nodes as the predictors and samples as leaves, we had a deeper dive into decision tree. Sicne single tree choose cutpoint that splitting the predictor space into two regions that leads to the greatest possible reduction in RSS, given by $\sum_{J}^{j=1}\sum_{i\in R_{j}}(y_{i}-\hat{y_{R_{j}}})^{2}$, where $\hat{y_{R_{j}}}$ is the mean prediction within the leaf, the more branch a tree has, the more precise the predcition would be. However, this deep-depth tree may overfit the training set and have high varriance among all trees. This is why we choose to fit ensemble methods, that can combine many trees and take the average of prediction made by each individual tree. The baseline ensemble methods we tried for the first step are Bagging, Randomforest, Extremely Randomized Trees and AdaBoost(they all have a package under 'sklearn.ensemble'). The strategy for bagging to reduce the variance is using boostrap to generate multiple training sets to fit multiple trees, and Random Forest further reduce the variance by randomly pick sub set of $\mathit{m}$ (usually $\sqrt{p}$) predictors from the total predictors(the number is $\mathit{p}$) at each split to make, so that all the trees would have less similar structure compared with bagging trees. Extremely Randomized Trees even futher reduce the variance by select predictor from the $\mathit{m}$ predictors totaly at random(rather than prefer those have higher split effciency). Adaboost takes a different strategy, which is to use a linear combination of multiple simple trees to reduce the residual step by step (essentially focus more on bias). To reduce the computaion load, we fit the baseline models on the 31 features after filtering. We splitted 20\% of the training set as validation to test MSE generated by each model and found the performance of Bagging, Randomforest, Extremely Randomized Trees are quit similar to each other, while Adaboost is much weaker. The MSE of the 4 models are as below: 

figure here(still running)

This is essentially because Adaboost focus more on reducing bias and the other 3 methods focus on reducing varriance. From this we can draw conclusion that, we can focus more on tuning those hyper parameters that can efficiently reduce variance of the model. 

Then we select random forest as main model to tune, and we use 5-fold cross validation to tune number of estimators (i.e. the total number of trees to take average on, which is key to reduce variance) on the newly generated predictor space with 2048 features within 1/7 of the total sample. Then we found that number of estimators=13 has the reasonable performance in this test. Here we met up with a variance-computaional complexity trade-off, that is: the more estimators we use, the less varriance of random forest suffers from. Given the computational resources we have(fitting 1 random forest on full data set takes more than 5 hours), and given the total sample size is 7 times of the test set we tune the model, setting the number of estimators to be 150 is safe and reasonable. Since the sample size is much larger than feature size, even though the max depth equal to total number of features, each leaf still contain more than 500 samples, we didn't put computational resources on tuning max depth of each tree.

We also plot the distribution of importance of predictors based on the baseline random forest, the result is as below:

figure here(upload later)

We can see most portion of predictors have little importance, but combining them together would yield good precision on validtion set($R_\textrm{RF_{new}^2 = 0.461 (\textrm{MSE}_\textrm{RF_{new} = 0.089)$). This made us explore more on Deep learning method, which considers the interaction among features.

\item Deep learning: Since the predictors describe a broad set of molecular properties, it is natural to expect non-linear interactions between them to play a role in determining the HOMO--LUMO gap. For this reason we decided to explore the training of simple neural networks that would serve to model these interactions and got acquainted with the basics of Deep Learning as a tool to tackle this problem. In all cases, we used an architecture of one or more dense (fully-connected) hidden layers between the input layer of $N$ parameters and the output layer of 1 node for the response we wanted to model. For each node in the hidden layers, we used the widely used ``relu'' function ($f(x \le 0) = 0, f(x > 0) = x$) as activation function. We trained the model using the \emph{keras} module in Python, using the \emph{adam} optimizer for the learning rate, and setting aside 20\% of training data for validation. The model would be fitting for the weights of the connections between pairs of nodes in the network, which in the case of 1 hidden layer with $M$ nodes, would result in $(N + 1) \times M$ parameters to be trained.

Taking a trial-and-error approach to setting the network architecture, we show the results of validation set MSE for 3 broad cases, using the 31 expressed features in the sample training dataset: 1) 1 hidden layer with variable number of nodes, 2) 2 hidden layers, with 31 nodes on the first and variable number of nodes on the second, and 3) 3 hidden layers, with 31 nodes on each of the first two, and variable number of nodes on the third.

\emph{Figure with 3 plots for 1, 2 and 3 hidden layers}

The result of this experiment showed that the MSE was not as sensitive to the addition of a second and third layer, as it was to the number of nodes in the first layer of the 1-hidden layer architecture. Given the computational limitations that were expected for a much larger set of features, we used this result to use the feature-engineered set to train only 1-layer models with a limited number of nodes in the same range as the number of predictors.


\end{enumerate}

\end{enumerate}


{\itshape
How did you tackle the problem? Credit will be given for:

  \begin{itemize}
  \item Diving deeply into a method (rather than just trying
    off-the-shelf tools with default settings). This can mean 
    providing mathematical descriptions or pseudo-code.
  \item Making tuning and configuration decisions using thoughtful experimentation.  
    This can mean carefully describing features added or hyperparameters tuned.
  \item Exploring several methods. This can contrasting two approaches
    or perhaps going beyond those we discussed in class.
  \end{itemize}

  \noindent Thoughtfully iterating on approaches is key.
  If you used existing packages or referred to papers or blogs for ideas,
  you should cite these in your report. 

  \begin{table}
    \centering
    \begin{tabular}{@{}lll@{}}
%      \toprule
      &\multicolumn{2}{c}{Mention Features  } \\
      & Feature & Value Set\\
      \midrule
      & Mention Head & $\mcV$ \\
      & Mention First Word & $\mcV$ \\
      & Mention Last Word & $\mcV$ \\
      & Word Preceding Mention & $\mcV$ \\
      & Word Following Mention & $\mcV$\\
      & \# Words in Mention & $\{1, 2, \ldots \}$ \\
      & Mention Type & $\mathcal{T}$ \\
      \bottomrule
      
    \end{tabular}
    \caption{Feature lists are a good way of illustrating problem specific tuning.}
  \end{table}

}

\section{Results}
This section should report on the following questions: 

(Antonio): Main result here should be a box plot with the MSE of different methods we tried. Models for the original data (256 features) should be different from models for the dataset of all the features.

{\itshape
\begin{itemize}
\item Did you create and submit a set of
  predictions? 
  

\item  Did your methods give reasonable performance?  
\end{itemize}

\noindent You must have \textit{at least one plot or table}
that details the performances of different methods tried. 
Credit will be given for quantitatively reporting (with clearly
labeled and captioned figures and/or tables) on the performance of the
methods you tried compared to your baselines.



\begin{table}
\centering
\begin{tabular}{llr}
 \toprule
 Model &  & Acc. \\
 \midrule
 \textsc{Baseline 1} & & 0.45\\
 \textsc{Baseline 2} & & 2.59 \\
 \textsc{Model 1} & & 10.59  \\
 \textsc{Model 2} & &13.42 \\
 \textsc{Model 3} & & 7.49\\
 \bottomrule
\end{tabular}
\caption{\label{tab:results} Result tables can compactly illustrate absolute performance, but a plot may be more effective at illustrating a trend.}
\end{table}
}



\section{Discussion} 

{\itshape
End your report by discussing the thought process behind your
analysis. This section does not need to be as technical as the others 
but should summarize why you took the approach that your did. Credit will be given for:

  \begin{itemize}
  \item Explaining the your reasoning for why you seqentially chose to
    try the approaches you did (i.e. what was it about your initial
    approach that made you try the next change?).  
  \item Explaining the results.  Did the adaptations you tried improve
    the results?  Why or why not?  Did you do additional tests to
    determine if your reasoning was correct?  
  \end{itemize}
 }

\end{document}

